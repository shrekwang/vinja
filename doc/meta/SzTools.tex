\documentclass[oneside,openany]{book}

  \usepackage{booktabs}
  \usepackage{mdframed}
  \usepackage{titlesec}
  \usepackage{float}
  \usepackage{fancyhdr}
  \usepackage{xcolor}
  \usepackage[bookmarks=true,bookmarksnumbered=true,bookmarksopen=true,pdftex,pdfstartview=FitH]{hyperref}
  \usepackage{xeCJK}
  %\setCJKmainfont{SimSun}
  \setCJKmainfont[BoldFont={SimHei}]{SimSun}

  \title{Vim-Sztools使用手册}
  \author{左晃右过 shrek.wang挨特gmail.com}

  \restylefloat{table}
  \pagestyle{fancy}
  \titleformat{\chapter}{\centering\Huge\bfseries}{第\,\thechapter\,章}{1em}{}

  \mdfdefinestyle{BestPracticeFrame}{
    linecolor=black,
    topline=false,
    rightline=false,
    leftline=false,
    bottomline=false,
    outerlinewidth=0pt,
    roundcorner=20pt,
    innertopmargin=\baselineskip,
    innerbottommargin=\baselineskip,
    innerrightmargin=10pt,
    innerleftmargin=10pt,
    backgroundcolor=gray!50!white}

  \mdfdefinestyle{SmallFrame}{
    linecolor=black,
    topline=false,
    rightline=false,
    leftline=false,
    bottomline=false,
    outerlinewidth=0pt,
    roundcorner=5pt,
    innertopmargin=5pt,
    innerbottommargin=5pt,
    innerrightmargin=3pt,
    innerleftmargin=3pt,
    backgroundcolor=gray!50!white}

\begin{document}
\maketitle

\chapter{简介}

  \begin{center}
    \large\textbf{工欲善其事,必先利其器}
  \end{center}

  Vim-Sztool是(又)一个vim的java开发插件, vim官网上已经有一个比较著名的java开发插件,Vim JDE, 
  另外eclim将eclipse作为后瑞,也是个很有特色的插件. 为什么还要再作一个插件呢
  \begin{enumerate}
    \item Vim JDE每次补全都需调用外部java命令用反射来读取类信息,速度比较慢.eclim因为用eclipse作为后端, 速度也是偏慢.
    \item 无法在vim中调试java程序.
  \end{enumerate}

  相比于现有的java开发类插件,Vim-Sztool有自己的一些特色
  \begin{enumerate}
    \item 快速的补全(类相关信息缓存在Agent中)
    \item java程序调试功能,可能是仅有的支持java调试的vim插件
    \item 扩展的NerdTree树,显示项目树,支持显示项目引用的源码包(source jar).
    \item 快速文件定位,类似于Eclipse中的open resource功能.
    \item 源码跳转,支持跳转到source jar包中的文件.
    \item 远程调试,可以连接到tomcat
    \item 直接从自带jdb中启动tomcat
  \end{enumerate}

\chapter{安装和配置}
  \section{需求}
  Vim-Sztool是一个用python,java,vim脚本混合编写的插件, 要使此插件能运行,系统需安装以下软件:
    \begin{itemize}
      \item vim7.3,只支持gvim版本
      \item Jdk1.6以上
      \item Python2.7 (2.6的也许可以运行),需安装pyparsing,BeautifulSoup,chardet模块
    \end{itemize}

  \section{安装}
    \begin{enumerate}
      \item 解压安装包, plugin目录覆盖到vim插件路径上,或
      者设置runtimepath, 比如set runtimepath+=C:/project/vim-sztool
      \item 在vimrc中添加全局变量 g:sztool\_home
      let g:sztool\_home="d:\\Program Files/vim/vimfiles/sztools"
      \item 设置环境变量
      JDK\_HOME,path(需要把gvim,python的可执行文件目录加到path中)
    \end{enumerate}
   如果安装成功,此时启动gvim,在系统托盘区会有个军刀的图标,这个是Agent程序,
   说明已经安装成功了.

  \section{配置}
    Vim-Sztool的默认配置文件位于/sztools/share/conf目录下,一般不要改动此文件中的内容,
  而是把文件复制到 ~/.sztools/目录下进行改动. Vim-Sztool会读取此两个目录下的内容,
  优先级以home目录下的为高.
  \begin{description}
    \item{sztools.cfg} 主要配置文件,用于配置jde的一些基本设置
    \item{vars.txt} 配置.classpath文件中引用的变量
    \item{stepfilters.txt} 配置jdb调试时,略过的一些package
    \item{db.conf} 配置Dbext用到的数据库连接
  \end{description}
  

\chapter{使用}
  \section{Shext}
  Shext是一个仿shell的东西,启动后, vim会被split成上下两个buffer. 上面一个是命令
编辑区, 用于编辑和运行命令. 下面一个是命令输出区,用于显示命令的输出.
  \newline
  Shext支持内置实现的一些的命令和系统的命令, 内置命令优先.
  \newline
  不支持交互式命令, 如ftp,mysql等 
  \newline
  \begin{mdframed}[style=BestPracticeFrame]
  最佳实践: 由于可能需要经常的切换window, 最好在vimrc中定义下列的map
    \begin{verbatim}
      map <C-j> <C-W>j
      map <C-k> <C-W>k
      map <C-h> <C-W>h
      map <C-l> <C-W>l
    \end{verbatim}
  \end{mdframed}

  Shext启动后, 如果输入"/",如果对应的参数是目路路径, 则在命令输出buffer会显示
对应路径的内容作为提示. 比如 
  \begin{verbatim} cd c:/ \end{verbatim}
 在"/"按下后, 会列出c:根目录的内容.(Shext默认在linux和win下都使用"/"作为目录分隔符)

  在命令编辑区, 在insert模式下输入"\$n;;", 则会把输出buffer的第n行补全到当前命令行中.
在某些条件下会比较有用.

  比如你可以先用 find --name *.java --text Apple 查找包含Apple的java文件, 在找到后,
如果想编辑其中的第5个文件, 则可以用 "edit \$5;;<cr>" 来实现. 注意这个补全是带状态的,
在\$5;;后,如果你还要编辑上一个文件,可以用<C-p>, 如果想编辑下一个文件,可以用<C-n>.

  \subsection{内置命令:}

  \begin{verbatim}
  改变和显示目录:
    pwd : 显示当前目录

    cd [arg] : 改变当前目录, 可使用bookmark和通配符
    举例
    cd ~/.vim/plugin
    cd work/project   --其中work是定义的一个书签,并非子目录
    cd *doc           --如果当前目录下只有一个doc结尾的子目录,则进入该子目录

    cdlist   : list the cd history.
    cdlist   : 列出cd过的目录历史.
    lsd      : 列出当前目录下的子目录.

    ls [ -l | -L ][ -t | -s | -n ] [ --help ] [args] 
    列当前目录内容. 
    -l 表示长格式, 每一行列一个文件
    -t 表示按时间排序
    -s 表示按大小排序
    -n 表示按名称排序
    --help 打印ls命令的帮助信息

    因为实现上的原因, 以"."开头的文件默认是不显示的, 你可以用 "ls ." 来显示所有文件


  书签 :
      bmadd   : 将当前目录加到bookmarks中,能在"cd"命令中引用
      bmedit  : 编辑"bookmarks"文件
      bmlist  : 列出所有bookmark.

  文件管理:
      touch [args]          : 新建文件或者更新文件时间戳
      rm  [args] [-r]       : 删除文件或目录
      mkdir [args]          : 新建目录
      rmdir [args]          : 删除空目录
      cp  [src...][dst]     : 复制文件或目录
      mv  [src...][dst]     : 移动文件或目录
      echo [args]           : 打印文本消息
      yank [args]           : clear "yankbuffer" and  yank selected file name the "yankbuffer".
      yank [args]           : 复制args表示的文件到"yankbuffer"(清空原"yankbuffer"内容)
      yankadd [args]        : 追加args表示的文件到"yankbuffer"
      cut  [args]           : 剪切args表示的文件到"yankbuffer"(清空原"yankbuffer"内容)
      cutadd [args]         : 剪切追加args表示的文件到"yankbuffer"

      paste                 : 粘贴"yankbuffer"中的文件名到当前目录.
                              如果是cut到"yankbuffer"的,同时删除源文件.

      yankbuffer            : 显示"yankbuffer"中的内容
      merge [src...][dst]   : 合并多个文件

      find [-n name][-t text] [-s size] [-p path] [--help] [args] : find files.

  File display & edit :
      cat [arg]      : 显示文件内容
      head [arg]     : 显示文件前10行
      edit [args]    : 在打开新tab,并编辑文件

  locate命令 :
      此命令在Locate一节描述

  杂项 :
      help           : 打印Shext帮助信息
      exit           : 退出Shext.
  \end{verbatim}

\section{Jdext}
  Jdext 是另一个 java 插件, 不同于流行插件Vjde, Jdext需要一个独立运行的Agent来完成补全,编译等功能。
这个Agent缓存了一些包名信息,并且采用了eclipse Jdt compiler来编译类,所以编译速度很快. 

\subsection{项目结构}
    Jdext使用了和eclipse一样的项目管理方式, 在项目根目中,有一个.classpath和.project文件. 所以建
议用eclipse建立项目,包括设置类路径等.在项目建立完成后, 再在Vim中进行开发.

   对于使用maven的项目来说, 还可以直接用maven命令行建maven项目,然后用
  \begin{verbatim} mvn eclipse:eclipse \end{verbatim}
  来生成eclipse项目文件. 此种方式生成的.classpath文件,引用的jar包都是以M2\_REPO来开头的, Vim-SzTool
在默认情况下找不到这个变量相对应的路径. 需要在~/.sztools/vars.txt配置变量对应的路径. 格式很简单
  \begin{verbatim} M2\_REPO = /usr/maven/repo \end{verbatim}
所以如果原来用eclipse建立了项目的,那么直接用:Jdext启动 jdext, 就可以使用补全,编译等功能了
如果是没用的, 可以启动:Jdext 后中,用:InitProject在当前目录生成 这两个文件. 类路径的在.classpath 
中,如果更改了这个文件,需要重启 vim和 Agent. 

   对于简单的类,如果只引用JDK自带类库的话, 也可以不用项目结构.在Jdext启动后,直接在任意目录新建Java
文件,然后就可以保存编译和调试.
  
   对于已经需要几个jar包的小工程, 也可以新建空目录后, 然后cd到空目录, 运行:InitProject命令来初始
化项目结构. 这个命令会生成.classpath和 src,lib目录. 然后你可以拷自己的jar包到lib目录, 并自己编辑
.classpath来更新类路径.

\subsection{启动Jdext}
   有几种方式可以启动Jdext:
    \begin{enumerate}
      \item 调用:Jdext 启动, 此命令会设置很多的map,比如(保存时自动编译文件等). 同时如果当前目录是
  在一个项目文件夹下(往上查找到.classpath文件),则Agent程序会在后台缓存类信息, 以加快补全速度.
      \item 在Shext中运行jde start命令, 功能同上.
    \end{enumerate}

\subsection{补全}
    Vim-Sztool自定了java类型的omni的补全,编辑 java时, 用 <ctrl-x><ctrl-o> 来进行.
    \begin{mdframed}[style=BestPracticeFrame]
      最佳实践: 安装SuperTab插件, 可以用tab键来做补全
      Vim-Sztool会做侦测,如果已经安装SuperTab插件, 则对于java类型的
      文件,默认的tab补全即是omni补全.
    \end{mdframed}

    \begin{enumerate}
      \item 可以补全包名,类名,名量名和对象的成员方法和字段
      \item 方法和字段补全是 ignore case 的,且可以引用通配符,比如 aa.to*er, 可以匹配成 aa.toLower
      \item 如果是大写开头的名字,则自动在类路径中查找类名。比如即使没有import FileReader类, 打FileRe 可以补全成FileReader类 
    \end{enumerate}
    
\subsection{编译}
    保存 java 时自动编译,如果有错误,则会生成 quickfix 列表 <C-N>,<C-p>跳转到前一个或后一个错误 

\subsection{运行}
    使用 :Run 或 ",," 来 运行当前class 

\subsection{其他}
    对于代码中引用到的类,没在import语句中声明的,可以用 :AutoImport 一次性生成全部import语句 
    \newline
    \newline
    选中要生成 getter,setter的行(可以多行),<leader>gg 来生成 getter,setter 
    \newline
    \newline
    使用 :InitProject 在当前目录初始化一个 jdext 项目, 如果编辑的文件是eclipse项目下的文件,则不需要再建项目. 
    \newline

\section{Jdb}
    Jdb是插件实现的调试功能(跟jdk里自带的jdb程序没关系), 需要在Jdext启动后再调用. 启动后, 默认会 
    split两块buffer, 一块是jdb命令buffer, 一块是jdb命令的输出buffer. jdb命令的buffer也是普通buffer, 除了在回车时自动执行命令(类似于shext) 

    jdb的大步部份调试功能是在jdb的buffer中用命令实现的, 有个例外是加断点. 只要启动了
    jde模式, 在编辑java文件的时候, 用<leader>tb来为当前行切换断点.

  \subsection{运行和附加}
    
    通过run命令运行Class, 后面为classname和参数, 如classname省略, 则运行当前编辑的Class.
    \begin{mdframed}[style=SmallFrame]
    \begin{flushleft}
    >run org.fake.test.Main --options args
    \end{flushleft}
    \end{mdframed}

    attach命令运行远程连接到java程序进行调试, 要求已经运行的程序必需要jpda调试模式启动
    用此命令可以远程连接到比如tomcat之类的应用. port为远程调试端口号, 目前只能连接到本
    机启动的应用, 其他host暂不支持.
    \begin{mdframed}[style=SmallFrame]
    \begin{flushleft}
    >attach 8080              
    \end{flushleft}
    \end{mdframed}

    disconnect命令用于断开attach上去的连接, shutdown用于中止正在调试的程序, hide用于
    隐藏jdb的窗口, exit退出调试 
    \begin{mdframed}[style=SmallFrame]
     \begin{flushleft}
    >disconnect\newline
    >shutdown\newline
    >hide\newline
    >exit                      
    \end{flushleft}
    \end{mdframed}

  \subsection{检查变量}
    print(或eval)命令用于打印表达式的值. reftype命令用于打印表达式的类型,
  inspect命令打印表达式的各个成员的值. locals打印函数内所有本地量的值, fields
  用于打印当前对象的各个字段的值
    \begin{mdframed}[style=SmallFrame]
      \begin{flushleft}
      >print expression\newline
      >eval expression\newline
      >reftype expression\newline                      
      >inspect expression\newline                      
      >locals\newline                      
      >fields
      \end{flushleft}
    \end{mdframed}

  \subsection{跟踪}

    单步进入,单步跳过,单步跳出,恢复. 这个跟eclipse的功能差不多. 快捷键也是一样的,分
    别是<F5><F6><F7><F8>.
    \begin{mdframed}[style=SmallFrame]
      \begin{flushleft}
      >step\_into\newline
      >step\_over\newline
      >step\_return\newline
      >resume           
      \end{flushleft}
    \end{mdframed}

    

  \subsection{线程和堆栈}
  threads列出当前jvm的所有线程, thread命令用于切换当前线程(如果有线程是suspend状态的话),
  frames列出当前线程的所有调用栈. frame命令用于切换当前的栈桢.(print,eval等命令只能打印当前栈桢的变量值)
  breakpoints列出当前jvmf所有的断点, bpa添加条件断点, setvalue改变var值
    \begin{mdframed}[style=SmallFrame]
      \begin{flushleft}
      >threads\newline
      >thread threadId\newline
      >frames\newline
      >frame n\newline
      >breakpoints\newline
      >bpa classname lineNum condition-expression\newline
      >setvalue  varname value-expression\newline
      \end{flushleft}
    \end{mdframed}


\section{ProjectTree}

  ProjectTree从NerdTree借鉴了相当多的代码, 基本上我写这个东西就是因为
NerdTree有些地方还难以适合做java的项目树(比如源包jar包的显示), 所以我
用python又写了一套.

  ProjectTree有自己的一些特点:
  \begin{enumerate}
    \item 按树结构显示source jar 
    \item 打开的节点显示不同的颜色,一目了然看到哪些文件在编辑
    \item 更方便的文件复制,移动,重命名功能
    \item 标记多个节点,以进行复制
    \item 按目录递归关闭正在编辑的文件
  \end{enumerate}

  以下是ProjectTree 可用的功能.
  \newline
  打开和关闭节点: 在ProjectTree中打开和关闭节点是最常用的功能. 对于目录节结点来说,
打开和关闭指的是expand和collapse操作, 而对于文件节点,打开指编辑,关闭指关闭正在编辑的
文件buffer. \footnote{对于buffer,window在vim中的概念,请参考vim的帮助文档}
  \newline
  (在Nerdtree中对于文件节点是没有关闭操作的)
  \newline
  正常情况下,打开文件节点后,节点会以不同的颜色显示.
  \begin{table}[H]
  \centering
      \begin{tabular}{p{40pt}p{220pt}}
        \toprule
        按键& 功能\\
        \midrule
          ?     &打印help信息\\
          <cr>  &打开选中节点\\
          o     &打开选中节点\\
          O     &递归打开选中节点\\
          t     &在新tab页中打开选中节点\\
          i     &在新window中打开选中节点\\
          go    &打开选中节点,光标停留在ProjectTree\\
          r     &刷新选中节点\\
          x     &关闭父节点\\
          s     &prompt to filter display node\\
          z     &关闭正在编辑的文件buffer, 如果当前是目录节点,则关闭目录下所有在编辑中的文件\\
          Z     &同上, 但是强行关闭文件,即使文件已改动\\
      \bottomrule
      \end{tabular}
  \end{table}

  由于vim本身的巨多的移动功能,在ProjectTree中扩展的移动不是很多. 对于已经打开的文件,
可以用"<",">"在这些节点之前快速移动.
  \begin{table}[H]
  \centering
      \begin{tabular}{p{40pt}p{220pt}}
        \toprule
        按键& 功能\\
        \midrule
          u     &光标移动到父节点\\
          m     &标记当前节点\\
          f     &在当前节点中查找字符串\\
          >     &光标移动到下一个编辑中的文件节点\\
          <     &光标移动到上一个编辑中的文件节点\\
      \bottomrule
      \end{tabular}
  \end{table}
  另外在编辑文件的buffer, 可以用ProjectTreeFind命令来定位左侧的
树结点.\newline 在我自己的vimrc中, 我做了如下映射
  \begin{mdframed}[style=SmallFrame]
    \begin{flushleft}
      nnoremap <silent> <F12> :ProjectTreeFind<cr>
    \end{flushleft}
  \end{mdframed}

  以下是一些删除剪切复制的功能,注意,这些操作是同步的,即在这些
粘贴删除实际完成前, 页面将无响应. 所以不要用使用这里的功能来
操作大的文件和目录,以避免可能的vim崩溃以导致数据丢失.
  \begin{table}[H]
  \caption{删除剪切复制等}
  \centering
      \begin{tabular}{p{40pt}p{220pt}}
        \toprule
        按键& 功能\\
        \midrule
          DD    &删除选中目录或文件\\
          Dm    &删除标记的目录或文件\\
          A     &新增目录或文件(如文件名以/结尾,则为目录)\\
          ya    &复制节点路径\\
          cc    &重命名当前节点\\
          yy    &复制当前文件或目录\\
          ym    &复制标记的文件或目录\\
          dd    &剪切当前文件或目录\\
          p     &粘贴复制或剪切的文件\\
          C     &更改树的根节点文件件\\
          B     &返回老的根节点\\
          U     &更改树的根节点为父目录\\
          QQ    &关闭ProjectTree\\
        \bottomrule
      \end{tabular}
  \end{table}

\section{Locate}

    Locate 功能有点像command-T插件, 用来快速定位文件的. 但是不同的是,这个功能需
要先对文件夹进行索引,索引后的 文件名信息存在sqlite的数据库中. 这样无论你的当前目
录 是在哪里, 都可以快速按文件名定位到已索引的目录中的文件.

\subsection{索引建立删除}
  索引管理索引需要在Shext中用locatedb命令管理(建索引时需要先cd到需要索引的目录) 
  \begin{itemize}
      \item locatedb add name : 建立索引,名称为name, 索引当前文件夹的内容
      \item locatedb remove name : 删除索引
      \item locatedb refresh name : 刷新索引
      \item locatedb list : 列出已建立的索引 
  \end{itemize}

\subsection{索引更新}
    当gvim启动后, 会有一个独立的Agent进程启动, 此进程会自动 
监视被索引目录的文件的新建和删除,并自动更新索引.如果文件在Agent
进程未启动条件下新建和删除, 可以手动执行locatedb refresh 命令更新

\subsection{调用}
   以 <leader>lw 来启动locate模式, 启动后,底下会有一个
 小的buffer用来显示匹配的文件名. 所有输入的字符都显示
 在command line上面, 并显示相应的匹配内容. 在匹配列表
 出来后,可用的功能如下
    \begin{itemize}
        \item <Esc> : 退出locate模式
        \item <CR>  : 打开(编辑)当前光标所在文件 
        \item <BS>  : 光标回退
        \item <C-j> : 光标下移
        \item <C-k> : 光标上移
        \item <C-v> : 复制剪贴板内容
        \item <C-b> : 在新buffer中打开文件
        \item <C-t> : 在新tab中打开文件
    \end{itemize}
    

\section{Dbext}

这个是vimsf 站点上的 dbext 的山寨版
使用前先确保安装了 pyodbc, 或 cx\_oracle 把数据库的连接参数写在 sztools home目录的 data目录下的 db.conf文件里

\subsection{配置}
   在使用前先确保在 ~/.sztools 目录下的db.conf中配置了想要连接的数据库.
   配置文件格式为
    \begin{mdframed}[style=SmallFrame]
    \begin{flushleft}
    servertype="mssql",host="127.0.0.1",user="sa",password="test"\newline
    servertype="mssql",host="127.0.0.1",user="sa",password="test"
    \end{flushleft}
    \end{mdframed}
    数据库名称不需要配置,在启动Dbext后用 use databasename 来切换.

\subsection{使用}
功能有限只能 :

    选中文本
    用 ,, 来执行,如果选中语句中有";", 默认会隔分多条来执行
    ld 用来列出所有的数据库

        在sql server连接时,如果不指定库,需要用use dbname来指定使用的库 

    lt 显示包含当前选中文本的所有表
    dt 显示表名为当前选中文本的数据表的表结构
    ,gs 把选中文本作为单条SQL执行, 即使包含";"也不分隔
    用 <c-x><c-o> 来补全 , 可以补全表名,字段名... (可以用通配符) 

   这个跟 dbext 对比 有一个好处是 数据用表格线分隔了, 列对得比较齐,输出如下格式 


\chapter{附录}
  以下是一些你可能会想要看一下的东西:
  \newline

  \href{http://www.vim.org/scripts/script.php?script\_id=1213}{http://www.vim.org/scripts/script.php?script\_id=1213} 
  Vjde,Vim插件
  \newline

  \href{http://eclim.org/}{http://eclim.org/} Eclim, vim插件,使用eclipse作为服务端
  \newline

  \href{http://vrapper.sourceforge.net/home/}{http://vrapper.sourceforge.net/home/}
  vrapper, clipse插件,免费
  \newline

  \href{http://www.viplugin.com/}{http://www.viplugin.com/}
  viplugin,eclipse插件,收费
  \newline

\chapter{后记}
    本文档用vim+latex写成. 感谢这两个伟大的工具
   \newline
   另外特别感谢Vim草堂的各位群友, 感谢YD的思凡和LE给我们每天带来的快乐
   
   
\end{document}
